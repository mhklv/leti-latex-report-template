\documentclass[a4paper]{article}
\usepackage{xltxtra}
\usepackage{graphicx}
\usepackage{fancybox}
\usepackage{fontawesome}
\usepackage{hyperref}
\usepackage{titletoc}
\usepackage{calc}
\usepackage{amsmath}
\usepackage{fancyhdr}
\usepackage{lipsum}
\usepackage{zref-savepos}
\usepackage{hypbmsec}
\usepackage{needspace}
\usepackage{xifthen}
\usepackage{listings}
\usepackage{float}
\usepackage{flafter}
\usepackage{placeins}
\usepackage{minted}
% \usepackage{lscape} % provides landscape environment
\usepackage{pdflscape} % provides landscape environment with rotation in a pdf file
\usepackage{enumitem}
\usepackage{array}
\usepackage{multirow}
\usepackage{longtable}
\usepackage{xtab}
\usepackage{accsupp}
\usepackage{booktabs}
\usepackage[table]{xcolor}

% \usepackage{newtxmath}
% \usepackage{fourier}
% \usepackage{palatino}




% Misc
\newcounter{fillparbox}
\newcommand{\fillparbox}[2][c]{%
  \stepcounter{fillparbox}% New \fillparbox
  \mbox{}\zsaveposx{\thefillparbox-L}\hfill\zsaveposx{\thefillparbox-R}% Mark Left and Right end-points remaining on line.
  \makebox[0pt][r]{\parbox[#1]{\dimexpr\zposx{\thefillparbox-R}sp-\zposx{\thefillparbox-L}sp}{\strut #2\strut}}%
}


% Шрифты и язык
\usepackage[base]{babel}
\usepackage{polyglossia}
\setmainlanguage{russian}
\setotherlanguage{english}
\setkeys{russian}{babelshorthands=true}
\defaultfontfeatures{Ligatures=TeX,Mapping=tex-text}

\setmainfont[Ligatures=Rare]{Times New Roman}
% \setmainfont{P052}
\setsansfont{Arial}
\setmonofont{DejaVu Sans Mono}
% \setmonofont{Latin Modern Mono Caps}
% \setmonofont{Mathematica7Mono}


% Automatic hyphenation rules switching
% \usepackage{microtype}
\usepackage{ucharclasses}
\setTransitionsForLatin{\begingroup\hyphenrules{english}}{\endgroup}


% Основной текст, заголовки
\newlength{\mainTextFontSize}
\newlength{\mainTextLeading}
\setlength{\mainTextFontSize}{14pt}
\setlength{\mainTextLeading}{\mainTextFontSize * \real{1.3}}

\setlength{\parindent}{1.25cm}
\setlength{\parskip}{1.75ex plus 0.5ex minus 0.3ex}
\widowpenalty=10000
\clubpenalty=10000

\newlength{\largeFontSize}
\newlength{\largishFontSize}
\newlength{\largeFontLeading}
\newlength{\largishFontLeading}
\setlength{\largeFontSize}{16pt}
\setlength{\largeFontLeading}{\largeFontSize * \real{1.15}}
\setlength{\largishFontSize}{15pt}
\setlength{\largishFontLeading}{\largishFontSize * \real{1.15}}

\newcounter{hOneCounter}
\newcounter{hTwoCounter}[hOneCounter]
\renewcommand{\thehTwoCounter}{\arabic{hOneCounter}.\arabic{hTwoCounter}}
\newcounter{hThreeCounter}[hTwoCounter]
\renewcommand{\thehThreeCounter}{\arabic{hOneCounter}.\arabic{hTwoCounter}.\arabic{hThreeCounter}}

\newcommand{\hzero}[1]{%
  \FloatBarrier
  \newpage
  \phantomsection
  \addcontentsline{toc}{section}{#1}
  \noindent{\centering\fontsize{\largeFontSize}{\largeFontLeading}\selectfont\sffamily\uppercase{#1}
    
  }
  \vspace{1ex}
}

\newcommand{\hone}[2][]{%
  \FloatBarrier
  \needspace{5\baselineskip}
  \refstepcounter{hOneCounter}
  \phantomsection
  \addcontentsline{toc}{subsection}{\texorpdfstring
    {\makebox[1.5em][l]{\arabic{hOneCounter}}#2}
    {\arabic{hOneCounter}~#2}
  }
  \vspace{4ex}
  \noindent{\centering\fontsize{\largeFontSize}{\largeFontLeading}\selectfont\sffamily\bfseries
    \arabic{hOneCounter}\hspace{0.8em}#2
    
  }
  \vspace{1ex}
  \ifthenelse{\equal{#1}{}}{}{\label{#1}}%
}

\newcommand{\htwo}[2][]{%
  \FloatBarrier
  \needspace{5\baselineskip}
  \refstepcounter{hTwoCounter}
  \phantomsection
  \addcontentsline{toc}{subsubsection}{\texorpdfstring
    {\makebox[2em][l]{\arabic{hOneCounter}.\arabic{hTwoCounter}}#2}
    {\arabic{hOneCounter}.\arabic{hTwoCounter}~#2}
  }
  \vspace{3ex}
  {\fontsize{\largishFontSize}{\largishFontLeading}\selectfont\sffamily\bfseries
    \arabic{hOneCounter}.\arabic{hTwoCounter}\hspace{0.8em}\fillparbox[t]{#2}
  }
  \vspace{0.75ex}
  \ifthenelse{\equal{#1}{}}{}{\label{#1}}%
}

\newcommand{\hthree}[2][]{%
  \FloatBarrier
  \needspace{5\baselineskip}
  \refstepcounter{hThreeCounter}
  \phantomsection
  \addcontentsline{toc}{paragraph}{\texorpdfstring
    {\makebox[2.75em][l]{\arabic{hOneCounter}.\arabic{hTwoCounter}.\arabic{hThreeCounter}}#2}
    {\arabic{hOneCounter}.\arabic{hTwoCounter}.\arabic{hThreeCounter}~#2}
  }
  \vspace{3ex}
  {\sffamily\bfseries
    \arabic{hOneCounter}.\arabic{hTwoCounter}.\arabic{hThreeCounter}\hspace{0.8em}\fillparbox[t]{#2}
  }
  \vspace{0.75ex}
  \ifthenelse{\equal{#1}{}}{}{\label{#1}}%
}


% Поля, нумерация страниц
\usepackage[top=2cm, bottom=2cm, left=3cm, right=1cm]{geometry}

\fancypagestyle{plain}{%
  \fancyhf{} % clear all header and footer fields
  \fancyfoot[C]{\leavevmode\BeginAccSupp{method=escape,ActualText={}}%
    \fontsize{\mainTextFontSize}{\mainTextLeading}\selectfont\mononums{\thepage}%
    \EndAccSupp{}%
  } % except the center
  \renewcommand{\headrulewidth}{0pt}
  \renewcommand{\footrulewidth}{0pt}
}
\pagestyle{plain}


% Специальные обозначения

% Mono emphasize for inline use
\makeatletter
\newcommand*\fsize{\dimexpr\f@size pt\relax}
\newcommand{\memph}[1]{\texttt{\fontsize{0.85\fsize}{0.85\fsize}\selectfont #1}}
\makeatother

\newcommand{\mononums}[1]{{\addfontfeatures{Numbers=Monospaced} #1}}

\urlstyle{same}


% Нумерованные и маркированные списки
% https://en.wikibooks.org/wiki/LaTeX/List_Structures
% http://ctan.altspu.ru/macros/latex/contrib/enumitem/enumitem.pdf

% \begin{itemize or enumerate}
% \item The first item
% \item The second item 
% \item The third 
% \end{itemize or enumerate}

\setlist{nosep, itemsep=0.5ex, left=0.5\parindent .. \parindent}


% Титульный лист (Тип работы, название работы, преподаватель, авторы)

% Тип работы
\newcommand{\letiTitleName}{Отчёт о лабораторной работе}

% Дисциплина
\newcommand{\letiTitleSubject}{Информатика}

% Тема
\newcommand{\letiTitleTopic}{Вомбаты}

% Авторы (группа, имена)
\newcommand{\letiTitleGroupNum}{0321}

\newcommand{\letiTitleAuthorOne}{М. Климанов}
\newcommand{\letiTitleAuthorTwo}{}
\newcommand{\letiTitleAuthorThree}{}
\newcommand{\letiTitleAuthorFour}{}


% Проверяющий (звание, имя)
\newcommand{\letiTitleProfTitle}{к.т.н.}
\newcommand{\letiTitleProfName}{тов. Миронов}


\renewcommand{\maketitle}{%
  \thispagestyle{empty}
  \newgeometry{left=3cm, right=3cm, top=2cm, bottom=2cm}
  {\centering
    \begin{minipage}{15cm}
      {\centering\fontsize{12pt}{12pt}\selectfont
        МИНОБРНАУКИ РОССИИ

        Федеральное государственное бюджетное образовательное учреждение
        
        высшего профессионального образования

        <<Санкт-Петербургский государственный электротехнический
        университет

        ,,ЛЭТИ`` им. В.И. Ульянова (Ленина)>>

        (СПбГЭТУ)
        
      }\vspace{0.5cm}\hrule
    \end{minipage}
    
  }
  \vfill
  {\centering
    \begin{minipage}{\textwidth}
      {\centering\fontsize{16pt}{16pt}\selectfont
        \setlength{\parskip}{1ex}
        \uppercase{Факультет компьютерных технологий и информатики}
        \vspace{0.7cm}
        
        {\bfseries Кафедра вычислительной техники}
        \vspace{0.7cm}
        
        {\bfseries \letiTitleName}
        \vspace{0.7cm}
        
        по дисциплине
        
        <<\letiTitleSubject>>
        \vspace{0.7cm}
        
        на тему
        
        {\bfseries <<\letiTitleTopic>>}
        
      }
    \end{minipage}
    
  }
  \vfill
  {\centering
    \begin{minipage}{\textwidth}
      {\fontsize{14pt}{14pt}\selectfont
        \setlength{\parskip}{2ex}
        {\bfseries \ifthenelse{\equal{\letiTitleAuthorTwo}{}}{Выполнил:}{Выполнили:}}
        
        \parbox[t]{0.35\textwidth}{%
          \ifthenelse{\equal{\letiTitleAuthorTwo}{}}{студент}{студенты}
          группы \letiTitleGroupNum
        }
        \hfill
        \begin{minipage}[t]{0.55\textwidth}
          {\setlength{\parskip}{1ex}
            \ifthenelse{\equal{\letiTitleAuthorOne}{}}{}{%
              \rule{0.35\textwidth}{0.4pt}
              \hfill
              \parbox[c]{0.6\textwidth}{\raggedleft
                (\letiTitleAuthorOne)
                \vspace{0.2cm}
              }
            }\par
            \ifthenelse{\equal{\letiTitleAuthorTwo}{}}{}{%
              \rule{0.35\textwidth}{0.4pt}
              \hfill
              \parbox[c]{0.6\textwidth}{\raggedleft
                (\letiTitleAuthorTwo)
                \vspace{0.2cm}
              }
            }\par
            \ifthenelse{\equal{\letiTitleAuthorThree}{}}{}{%
              \rule{0.35\textwidth}{0.4pt}
              \hfill
              \parbox[c]{0.6\textwidth}{\raggedleft
                (\letiTitleAuthorThree)
                \vspace{0.2cm}
              }
            }\par
            \ifthenelse{\equal{\letiTitleAuthorFour}{}}{}{%
              \rule{0.35\textwidth}{0.4pt}
              \hfill
              \parbox[c]{0.6\textwidth}{\raggedleft
                (\letiTitleAuthorFour)
                \vspace{0.2cm}
              }
            }\par
          }
        \end{minipage}
      }
    \end{minipage}
    \vfill
    \begin{minipage}{\textwidth}
      {\fontsize{14pt}{14pt}\selectfont
        \setlength{\parskip}{2ex}
        {\bfseries Проверил:}

        \parbox[c]{0.35\textwidth}{%
          \letiTitleProfTitle
        }
        \hfill
        \begin{minipage}[c]{0.55\textwidth}
          {\setlength{\parskip}{1ex}
            \rule{0.35\textwidth}{0.4pt}
            \hfill
            \parbox[c]{0.6\textwidth}{\raggedleft
              (\letiTitleProfName)
            }\par
          }
        \end{minipage}
      }
    \end{minipage}
    
  }
  \vfill
  {\centering\fontsize{14pt}{14pt}\selectfont
    Санкт-Петербург, \oldstylenums{\the\year{}}
    
  }
  \restoregeometry
  \newpage
}



% Содержание
\setcounter{tocdepth}{4}


% Рисунки с нумерацией и cross-references

% See figure \ref{fig:test}
% \letiFigure[fig:test]{width=\textwidth}{Вомбат}{img/2.png}

\newcounter{figureCounter}
\renewcommand{\thefigureCounter}{\arabic{figureCounter}}


% 1 (opt) -- label name
% 2 -- Размеры изображения
% 3 -- Название изображения
% 4 -- Файл изображения
\newcommand{\letiFigure}[4][]{%
  \begin{figure}
    \refstepcounter{figureCounter}
    \ifthenelse{\equal{#1}{}}{}{\label{#1}}
    {\centering
      \includegraphics[#2]{#4}

      \vspace{1ex}
      {\fontsize{12pt}{12pt}\selectfont\itshape
        Рисунок \thefigureCounter\ --- #3
      }
      
    }
  \end{figure}
}


% Формулы с нумерацией и cross-references
% https://en.wikibooks.org/wiki/LaTeX/Mathematics
% https://en.wikibooks.org/wiki/LaTeX/Advanced_Mathematics


% Таблицы с нумерацией и cross-references
% https://en.wikibooks.org/wiki/LaTeX/Tables
% https://texblog.org/2019/06/03/control-the-width-of-table-columns-tabular-in-latex/

\newcounter{tableCounter}
\renewcommand{\thetableCounter}{\arabic{tableCounter}}
\newcommand{\letiTableCaption}[1]{\fontsize{12pt}{12pt}\selectfont\itshape
  Таблица \thetableCounter\ --- #1
}

\usepackage[tableposition=below]{caption}
\captionsetup[longtable]{skip=0em,font=large}
\captionsetup[table]{skip=0em,font=large}

% \setlength{\tabcolsep}{10pt} % Horizontal spacing between columns. Default value: 6pt
% \renewcommand{\arraystretch}{1.5} % Spacing between rows. Default value: 1

% \begin{longtable}{ | m{0.2\textwidth} | c | c | }
%   \caption*{\textit{Таблица 1 --- Демонстрация}} \\
%   % \multicolumn{3}{l}{\parbox{9cm}{\textit{Таблица 1 --- Демонстрация}}} \\
%   \hline
%   \textbf{First entry} & \textbf{Second entry} & \textbf{Third entry} \\
%   \hline
%   \endfirsthead
%   % \multicolumn{3}{c}{\textit{Продолжение таблицы 1}} \\
%   \caption*{\textit{Продолжение таблицы 1}} \\
%   \hline\hline
%   \textbf{First entry} & \textbf{Second entry} & \textbf{Third entry} \\
%   \hline
%   \endhead
%   \hline\hline
%   % \multicolumn{3}{c}{\textit{Таблица 1 продолжена на следующей странице}} \\
%   \caption*{\textit{Таблица 1 продолжена на следующей странице}} \\
%   \endfoot
%   \endlastfoot
  
%   \hline
%   cell1 dummy text dummy  text dummy text& cell2 & cell3 \\ 
%   \hline
%   cell1 dummy text dummy text dummy text & cell5 & cell6 \\ 
%   \hline
%   cell7 & cell8 & cell9 \\
%   \hline
%   cell1 dummy text dummy  text dummy text& cell2 & cell3 \\ 
%   \hline
%   cell1 dummy text dummy text dummy text & cell5 & cell6 \\ 
%   \hline
%   cell7 & cell8 & cell9 \\
%   \hline
%   cell1 dummy text dummy  text dummy text& cell2 & cell3 \\ 
%   \hline
%   cell1 dummy text dummy text dummy text & cell5 & cell6 \\ 
%   \hline
%   cell7 & cell8 & cell9 \\
%   \hline
    %   \end{longtable}

% \begin{letiLongTable}[tab:dummy1]
%   {Демонстрация таблицы, которая автоматически разбивается на несколько страниц}
%   {| m{0.4\textwidth} c c |}
%   {\textbf{First entry} & \textbf{Second entry} & \textbf{Third entry} \\}
  
%   \hline
%   cell1 dummy text dummy  text dummy text& cell2 & cell3 \\ 
%   \hline
%   cell1 dummy text dummy text dummy text & cell5 & cell6 \\ 
%   \hline
% \end{letiLongTable}


% 1 (opt) -- label
% 2 -- название таблицы
% 3 -- table spec
% 4 -- строка заголовока таблицы
\newenvironment{letiLongTable}[4][]{%
  \refstepcounter{tableCounter}
  \ifthenelse{\equal{#1}{}}{}{\label{#1}}
  \begin{longtable}{#3}
    \caption*{\textit{Таблица \thetableCounter\ --- #2}} \\
    \hline
    #4
    \endfirsthead
    \caption*{\textit{Продолжение таблицы \thetableCounter}} \\
    \hline\hline
    #4
    \endhead
    \hline\hline
    \caption*{\textit{Таблица \thetableCounter\ продолжена на следующей странице}} \\
    \endfoot
    \endlastfoot}
    {
  \end{longtable}
}

% 1 (opt) -- label
% 2 -- название таблицы
% 3 -- table spec
\newenvironment{letiTable}[3][]{%
  \refstepcounter{tableCounter}
  \ifthenelse{\equal{#1}{}}{}{\label{#1}}
  \begin{table}
    \caption*{\textit{Таблица \thetableCounter\ --- #2}}
    \centering\begin{tabular}{#3}
    }
    {
  \end{tabular}\end{table}
}


% Листинги с нумерацией и cross-references
\usemintedstyle{trac}
\renewcommand{\theFancyVerbLine}{%
  \BeginAccSupp{method=escape,ActualText={}}%
  \textcolor[rgb]{0.5,0.5,0.5}%
  {\arabic{FancyVerbLine}}%
  \EndAccSupp{}%
}

\newcounter{listingCounter}
\renewcommand{\thelistingCounter}{\arabic{listingCounter}}

% 1 (opt) -- label name
% 2 -- Название листинга
% 3 -- Язык программирования
\newenvironment{letiListing}[3][]
{%
  \vspace{2ex}
  \noindent
  \refstepcounter{listingCounter}
  \ifthenelse{\equal{#1}{}}{}{\label{#1}}
    {\raggedleft\fontsize{12pt}{12pt}\selectfont\itshape
      Листинг \thelistingCounter\ --- #2\par
    }
  \vspace{-0.2cm}
  \fontsize{10pt}{10pt}\selectfont\VerbatimEnvironment
  \begin{minted}[mathescape,
    linenos,
    autogobble,
    xleftmargin=1cm,
    breaklines,
    frame=single,
    framesep=2mm]{#3}}
  {\end{minted}\vspace{2ex}}

% 1 (opt) -- label name
% 2 -- Название листинга
% 3 -- Язык программирования
% 4 -- file
\newcommand{\letiListingInput}[4][]
{%
  \needspace{5\baselineskip}
  \vspace{2ex}
  \noindent
  \refstepcounter{listingCounter}
  \ifthenelse{\equal{#1}{}}{}{\label{#1}}
    {\raggedleft\fontsize{12pt}{12pt}\selectfont\itshape
      Листинг \thelistingCounter\ --- #2\par
    }
    \vspace{-0.2cm}
  \fontsize{10pt}{10pt}\selectfont
  \inputminted[mathescape,
    linenos,
    autogobble,
    xleftmargin=1cm,
    breaklines,
    frame=single,
    framesep=2mm]{#3}{#4}
}


% Возможность добавлять горизонтальные страницы
% https://tex.stackexchange.com/questions/278113/single-landscape-page-with-page-number-at-the-bottom
% \begin{landscape}
%   ...
% \end{landscape}


% Возможность добавлять приложения и cross-references
\newcounter{appendixCounter}
\renewcommand{\theappendixCounter}{\Asbuk{appendixCounter}}

\newcommand{\hAppendix}[2][]{%
  \FloatBarrier
  \newpage
  \refstepcounter{appendixCounter}
  \phantomsection
  \addcontentsline{toc}{section}{Приложение \theappendixCounter: #2}
  \hspace{0pt}
  \vfill

  \noindent{\centering\fontsize{\largeFontSize}{\largeFontLeading}\selectfont
    \sffamily\uppercase{Приложение \theappendixCounter}

    \bfseries #2
    
  }
  \ifthenelse{\equal{#1}{}}{}{\label{#1}}
  \vfill
  \hspace{0pt}
  \newpage
  \setcounter{tableCounter}{0}
  \renewcommand{\thetableCounter}{\theappendixCounter.\arabic{tableCounter}}
  \setcounter{figureCounter}{0}
  \renewcommand{\thefigureCounter}{\theappendixCounter.\arabic{figureCounter}}
  \setcounter{listingCounter}{0}
  \renewcommand{\thelistingCounter}{\theappendixCounter.\arabic{listingCounter}}
}


% Возможность добавлять сноски
% \footnote{I am a footnote}
\renewcommand{\footnotesize}{\fontsize{10pt}{10pt}\selectfont}


% Библиография
% https://en.wikibooks.org/wiki/LaTeX/Bibliography_Management

\usepackage{etoolbox}
\patchcmd{\thebibliography}{\section*{\refname}}{}{}{}

% See \cite[с. 200--220]{lamport94}

% \begin{thebibliography}{9}
% \bibitem{lamport94}
%   Leslie Lamport,
%   \textit{\LaTeX: a document preparation system},
%   Addison Wesley, Massachusetts,
%   2nd edition,
%   1994.
% \end{thebibliography}


% Commands that must be executed in document block
\newcommand{\letiPreamble}{
  \fontsize{\mainTextFontSize}{\mainTextLeading}\selectfont
  \renewcommand{\contentsname}{\noindent\centering
    \textnormal{\fontsize{16pt}{16pt}\selectfont\sffamily\uppercase{Содержание}}}
}


% #######################################################################
% #######################################################################
% #######################################################################


% Тип работы
\renewcommand{\letiTitleName}{Отчёт о лабораторной работе №\,2}

% Дисциплина
\renewcommand{\letiTitleSubject}{Программирование}

% Тема
\renewcommand{\letiTitleTopic}{Вычисление суммы ряда}

% Авторы (группа, имена)
\renewcommand{\letiTitleGroupNum}{0321}

\renewcommand{\letiTitleAuthorOne}{Климанов\,М.}
\renewcommand{\letiTitleAuthorTwo}{Русских\,В.}
\renewcommand{\letiTitleAuthorThree}{Феллер\,В.}
\renewcommand{\letiTitleAuthorFour}{}


% Проверяющий (звание, имя)
\renewcommand{\letiTitleProfTitle}{Преподаватель каф. ВТ, к.т.н., доцент}
\renewcommand{\letiTitleProfName}{Миронов\,С.Э.}


% Метаданные PDF
\hypersetup{%
  pdfauthor={Климанов М., 0321},
  pdftitle={\letiTitleTopic},
  pdfsubject={\letiTitleName},
  pdfkeywords={ЛЭТИ},
  pdfproducer={XeLaTeX, hyperref},
  pdfcreator={LaTeX}
}


% In case of overfull boxes
% \setlength{\emergencystretch}{4em}
% \tolerance=1000

\usepackage{dcolumn}
\newcolumntype{d}[1]{D{,}{,}{#1} }


\renewcommand{\epsilon}{\varepsilon}
\begin{document}
\maketitle
\letiPreamble
\mononums{\tableofcontents}
\newpage

\hone{Цель работы}

Целью этой лабораторной работы является изучение математических и
циклических операторов языка \memph{C} и получение практических
навыков создания программ для эффективного вычисления суммы ряда.


\hone{Задание}

Дан функциональный числовой ряд
\begin{equation}\label{eq:theSeries}
  S(x) = \sum_{m = 1}^\infty (-1)^{m + 1} \cdot \frac{2^{m - 1}}{x^m} =
  \frac{1}{x} - \frac{2}{x^2} + \frac{4}{x^3} - \frac{8}{x^4} + \ldots
\end{equation}

Необходимо определить его область сходимости и написать программу на
языке \memph{C}, которая эффективно вычисляет его сумму в заданной
точке с заданной допустимой погрешностью $\pm\epsilon$. Иными словами
нужно вычислить такую частичную сумму
$S_n(x) = \sum_{m = 1}^{n} (-1)^{m + 1} \cdot \frac{2^{m -
    1}}{x^m}$, что
\begin{equation}
  |S(x) - S_n(x)| < \epsilon
\end{equation}


\newpage\hone{Ход работы}

В ходе выполнения лабораторной работы сначала была написана наивная
реализация алгоритма вычисления суммы ряда \eqref{eq:theSeries}. После
этого последовательно были найдены две возможности вычислить сумму
этого ряда более эффективно. Исходный код всех трёх реализаций
приведён в приложении \ref{app:code}.


\htwo{Сходимость ряда}

Область сходимости ряда \eqref{eq:theSeries} можно найти с помощью
радикального признака сходимости Коши \cite[с. 298]{fichtenholz}:
\begin{equation}
  \lim_{n \to \infty} \sqrt[n]{\frac{2^{n-1}}{|x^n|}} =
  \lim_{n \to \infty} \frac{2}{\sqrt[n]{2} \cdot |x|} = \frac{2}{|x|} < 1
  \implies x \in (-\infty ; -2) \cup (2 ; +\infty)
\end{equation}

Таким образом, ряд \eqref{eq:theSeries} сходится абсолютно на области
$(-\infty ; -2) \cup (2 ; +\infty)$. Когда программе на вход подаётся
значение $x$, не входящее в эту область, она должна завершаться и
сообщать об ошибке.


\htwo{Наивный подход}

Известно, что остаток знакочередующегося ряда не превышает по модулю
первого отброшенного слагаемого \cite[с. 330]{fichtenholz}. Этот факт
позволяет использовать следующий критерий остановки алгоритма:
\emph{алгоритм должен прекратить суммирование, когда последнее
  вычисленное слагаемое по модулю меньше $\epsilon$.}

Наивная реализация приведена в листинге \ref{lst:naive}. Из этого
листинга видно, что на каждой итерации рабочего цикла программы
используется 1 операция деления, 2 операции умножения, 1 операция
сложения, операция сравнения, условный оператор и вызов функции
\memph{fabs()}. В следующей версии программы количество операций в
цикле значительно сократится.


\htwo{Схема Горнера}

Чтобы исключить операцию сравнения и условный оператор на каждой
итерации цикла, можно заранее вычислить такое $n$, что
$|S(x) - S_n(x)| < \epsilon$. Тогда количество итераций рабочего цикла
будет известно на момент его начала и лишние сравнения не понадобятся.

Сначала нужно найти такое $n$, что $|a_{n+1}| < \epsilon$,
преобразовав это неравенство:
\begin{equation}\label{eq:horNDef}
  |a_{n + 1}| = \left|\frac{2^n}{x^{n+1}}\right| = \frac{2^n}{|x|^{n+1}} < \epsilon
  \implies n \ge \frac{\ln{2\epsilon}}{\ln{\frac{2}{|x|}}}
\end{equation}
Теперь, если $|S(x) - S_n(x)| < |a_{n+1}|$ согласно оценке остатка
знакочередующегося ряда, и $|a_{n+1}| < \epsilon$, то
$|S(x) - S_n(x)| < \epsilon$ при любом найденном в \eqref{eq:horNDef}
$n$, что и требовалось получить.

Если $x < -2$, то
$S(x) = -\frac{1}{|x|} - \frac{2}{|x|^2} - \frac{4}{|x|^3} - \ldots$
То есть, ряд \eqref{eq:theSeries} перестаёт быть
знакочередующимся. Это значит, что оценка $n$ в \eqref{eq:horNDef} в
таком случае некорректна и необходимо вычислять $n$ по-другому.

Чтобы получить оценку $n$ для случая $x < -2$, можно вычислить остаток
ряда
\begin{equation}\label{eq:majoringSeries}
  R(x) = \sum_{m = 1}^{\infty} \left( \frac{2}{|x|} \right)^m
\end{equation}
Остаток ряда \eqref{eq:majoringSeries} будет больше остатка ряда
\eqref{eq:theSeries} при $x < -2$, т.к.
$\left( \frac{2}{|x|} \right)^m > \frac{2^{m-1}}{|x|^m}$ для всех m.
Остаток ряда \eqref{eq:majoringSeries} можно вычислить точно как остаток геометрического ряда:
\begin{equation}
  R(x) - R_n(x) = \frac{\frac{2}{|x|} \left( 1 - \left(\frac{2}{|x|}\right)^n \right) }{1 - \frac{2}{|x|}} - \frac{\frac{2}{|x|}}{1 - \frac{2}{|x|}} = \frac{2^{n+1}}{|x|^n (|x| - 2)}
\end{equation}
Новую оценку $n$ можно получить аналогично \eqref{eq:horNDef},
преобразовав неравенство:
\begin{equation}
  \frac{2^{n+1}}{|x|^n (|x| - 2)} < \epsilon \implies n > \frac{\ln{\frac{\epsilon}{2}} + \ln(|x| - 2)}{\ln{\frac{2}{|x|}}}
\end{equation}

$S_n(x)$ можно вычислять ещё более эффективно, преобразовав его по
схеме Горнера \cite[с. 538]{knuth2} следующим образом:
\begin{align}
  &S_n(x) = \frac{(-1)^2}{2} \cdot \frac{2}{x} + \frac{(-1)^3}{2} \cdot \left(\frac{2}{x}\right)^2 + \ldots + \frac{(-1)^{n+1}}{2} \cdot \left(\frac{2}{x}\right)^{n} = \nonumber\\
  &= \left(\rule{0cm}{32pt}\left(\rule{0cm}{24pt} \ldots \left( \frac{2}{x} \cdot \frac{(-1)^{n+1}}{2} + \frac{(-1)^n}{2} \right) \frac{2}{x} + \ldots + \frac{(-1)^3}{2} \right) \frac{2}{x} + \frac{(-1)^2}{2} \right) \frac{2}{x}
\end{align}

Реализация вычисления суммы ряда с разложением по схеме Горнера приведена в
листинге \ref{lst:horner}. Здесь на каждой итерации рабочего цикла
используется только 1 умножение и 2 сложения.


\htwo{Нахождение функции, в которую раскладывается ряд}

Поскольку ряд \eqref{eq:theSeries} сходится абсолютно, его слагаемые
можно произвольно переставлять, и сумма ряда при этом не
изменится. Его можно представить как разность двух геометрических
рядов и получить очень простую элементарную функцию, в которую
раскладывается ряд:
\begin{align}
  S(x) &= \sum_{m = 1}^{\infty} (-1)^{m+1} \cdot \frac{2^{m-1}}{x^m}
         = \frac{1}{2} \left( \sum_{m = 1}^{\infty} \left( \frac{2}{x} \right)^{2m-1} - \sum_{m = 1}^{\infty} \left( \frac{2}{x} \right)^{2m} \right) \nonumber\\
         &= \frac{1}{x + 2}
\end{align}

Реализация вычисления суммы ряда с использованием функции
$\frac{1}{x+2}$ приведена в листинге \ref{lst:constant-time}. Здесь
сумма ряда вычисляется с точностью, которую могут дать числа с
плавающей точкой двойной точности (обычно около 15 десятичных цифр
после запятой).


\hone{Тестирование программ}

В этом разделе будет проведена проверка корректности работы созданных
программ путём запуска тест-кейсов для каждого варианта реализации. В
начале будет протестирована самая точная реализация, которая будет
использована как эталон для обнаружения недостатков в других
реализациях.


\htwo{Тестирование абсолютно точной реализации}

В таблице \ref{tab:precise} приведены входные значения $x$ и
соответствующие им выходные значения $S(x)$, произведённые программой
из листинга \ref{lst:constant-time}.

\setlength{\tabcolsep}{10pt}
\renewcommand{\arraystretch}{1.3}
\begin{letiTable}[tab:precise]{Результаты тестов программы \ref{lst:constant-time}}{d{3} d{6}}
  \toprule
  \multicolumn{1}{c}{$x$} & \multicolumn{1}{c}{$S(x)$} \\
  \midrule
  -10 & -0,125 \\
  -3 & -1 \\
  -2,001 & -1000 \\
  2,01 & 0,249377 \\
  4 & 0,166667 \\
  8 & 0,1 \\
  \bottomrule
\end{letiTable}


\htwo{Тестирование наивной реализации}

В таблице \ref{tab:naive} приведены входные значения $x$ и $\epsilon$,
а также соответствующие им выходные значения $n$ и $S_n(x)$,
произведённые программой из листинга \ref{lst:naive}. С целью оценки
погрешности в ней также приведены остатки частичной суммы ряда. Серым
цветом в этой таблице выделены тест-кейсы, в которых остаток превысил
заданное значение $\epsilon$.

Как уже было замечено, оценка остатка с помощью последнего
отброшенного члена ряда некорректна для отрицательных $x$. Именно
поэтому в строке 3 реальный остаток суммы ряда превысил заданный
$\epsilon$.

Чем ближе $x$ к точкам $-2$ и $2$ при фиксированном $\epsilon$, тем
медленнее убывают слагаемые в ряде и тем больше членов ряда необходимо
вычислить, чтобы получить суму ряда с заданной точностью. Например, в
строке 4 $n = 1024$. Это значит, что программе необходимо вычислить
1024-й член ряда: $\frac{2^{1023}}{x^{1024}}$. При этом максимально
допустимое значение показателя в числах с плавающей точкой двойной
точности равно 1023. Как следствие, происходит переполнение числа с
плавающей точкой и расчёты завершаются значительно раньше, чем
требуется для достижения нужной точности.


\begin{letiTable}[tab:naive]{Результаты тестов программы \ref{lst:naive}}{d{3} d{5} d{0} d{6} d{6}}
  \toprule
  \multicolumn{1}{c}{$x$} & \multicolumn{1}{c}{$\epsilon$} &
  \multicolumn{1}{c}{$n$} & \multicolumn{1}{c}{$S_n(x)$} &
  \multicolumn{1}{c}{$|S(x) - S_n(x)|$} \\
  \midrule
  -10 & 0,1 & 2 & -0,12 & 0,005 \\
  -10 & 0,01 & 3 & -0,124 & 0,001 \\
  \rowcolor[gray]{0.9} -3 & 0,0001 & 22 & -0,999866 & 0,000134 \\
  \rowcolor[gray]{0.9} -2,001 & 0,001 & 1024 & -400,327837 & 599,672163 \\
  2,01 & 0,01 & 785 & 0,254348 & 0,004971 \\ 
  4 & 0,0001 & 13 & 0,166687 & 0,00002 \\
  8 & 0,00001 & 8 & 0,099998 & 0,000002 \\
  \bottomrule
\end{letiTable}


\htwo{Тестирование реализации по схеме Горнера}

В таблице \ref{tab:horner} приведены тест-кейсы для программы
\ref{lst:horner}. Из сравнения таблиц \ref{tab:naive} и
\ref{tab:horner} видно, что недостатки, характерные для наивной
реализации, исправлены в реализации по схеме Горнера.

\begin{letiTable}[tab:horner]{Результаты тестов программы \ref{lst:horner}}{d{3} d{5} d{0} d{6} d{6}}
  \toprule
  \multicolumn{1}{c}{$x$} & \multicolumn{1}{c}{$\epsilon$} &
  \multicolumn{1}{c}{$n$} & \multicolumn{1}{c}{$S_n(x)$} &
  \multicolumn{1}{c}{$|S(x) - S_n(x)|$} \\
  \midrule
  -10 & 0,1 & 1 & -0,1 & 0,025 \\
  -10 & 0,01 & 3 & -0,124 & 0,001 \\
  -3 & 0,0001 & 25 & -0,999960 & 0,00004 \\
  -2,001 & 0,001 & 29025 & -999,999500 & 0,0005 \\
  2,01 & 0,01 & 785 & 0,254348 & 0,004971 \\ 
  4 & 0,0001 & 13 & 0,166687 & 0,00002 \\
  8 & 0,00001 & 8 & 0,099998 & 0,000002 \\
  \bottomrule
\end{letiTable}


\newpage\hone{Выводы}

В ходе лабораторной работы изучены математические и циклические
операторов языка \memph{C}, написаны три программы, вычисляющие один и
тот же ряд с разной степенью эффективности. Изучены способы
оптимизации вычисления многочленов (схемы Горнера и
Эстрина). Использованные техники оптимизации вычислений можно
использовать для более широкого класса задач (вычисление других рядов,
сумм, многочленов).


\hzero{Список использованных источников}

\begin{thebibliography}{9}
\bibitem{fichtenholz} Фихтенгольц\,Г.М. Курс дифференциального и
  интегрального исчисления. В 3 т. Т. II / Пред. и
  прим. А.А. Флоринского. : 8-е изд. --- М. : ФИЗМАТЛИТ, 2003. ISBN
  5-9221-0157-9.

\bibitem{knuth2} Кнут, Дональд Эрвин. Искусство программирования,
  т.\,2. Получисленные алгоритмы, 3-е изд. : Пер. с англ. --- М. : ООО
  <<И.Д. Вильямс>>, 2017. ISBN 978-5-8459-0081-4 (рус.).
  
\end{thebibliography}


\hAppendix[app:code]{Исходный код программ}

\begin{letiListing}[lst:naive]{Наивное вычисление суммы ряда}{c}
#include <stdio.h>
#include <math.h>


int main() {
    int MAX_SERIES_N = 10000; // Максимальный номер члена ряда.
    int i;
    double eps;               // epsilon -- такое число, что остаток ряда не должен его превысить.
    double x, initial_x, series_sum = 0, series_term, numerator = 1;

    printf("S = (1 / x) - (2 / x^2) + (4 / x^3) - (8 / x^4) + ...\n");
    printf("x = ");
    scanf("%lf", &x);

    if (x >= -2.0 && x <= 2.0) {
        // Ряд расходится, программа завершается.
        printf("\nРяд расходится в заданной точке x.\n");
        return 0;
    }
    initial_x = x;
    
    printf("epsilon = ");
    scanf("%lf", &eps);

    if (eps <= 0) {
        // Некорректный epsion
        printf("\nEpsilon -- это требуемый остаток ряда. Он должен быть положительным.\n");
        return 0;
    }
    
    printf("\n...Вычисления...\n");

    for (i = 1; i <= MAX_SERIES_N; ++i) {
        series_term = numerator / x;
        series_sum += series_term;
        numerator *= -2;                   // Минус обеспечивает чередование знаков.
        x *= initial_x;

        if (fabs(series_term) < eps) {
            printf("Член ряда меньше epsilon. Остановка вычислений...\n");
            ++i;
            break;
        }
    }

    printf("Номер последнего вычисленного элемента ряда: %d\n", i - 1);
    printf("Частичная сумма ряда S_%d = %lf\n", i - 1, series_sum);

    return 0;
}
\end{letiListing}  

\newpage\begin{letiListing}[lst:horner]{Вычисление суммы ряда по схеме Горнера}{c}
#include <stdio.h>
#include <math.h>


int main() {
    int MAX_SERIES_N = 1000000; // Максимальный номер члена ряда.
    int i, n;
    double eps;                 // epsilon -- такое число, что остаток ряда не должен его превысить.
    double x, a, b;

    printf("S = (1 / x) - (2 / x^2) + (4 / x^3) - (8 / x^4) + ...\n");
    printf("x = ");
    scanf("%lf", &x);

    if (x >= -2.0 && x <= 2.0) {
        // Ряд расходится, программа завершается.
        printf("\nРяд расходится в заданной точке x.\n");
        return 0;
    }
    
    printf("epsilon = ");
    scanf("%lf", &eps);

    if (eps <= 0) {
        // Некорректный epsion
        printf("\nEpsilon -- это требуемый остаток ряда. Он должен быть положительным.\n");
        return 0;
    }
    
    printf("\n...Вычисления...\n");

    if (x > 2) {
        n = (int) (log(2.0 * eps) / log(2.0 / fabs(x))) + 1;
    }
    else {
        // При отрицательных x ряд больше не знакочередующийся, а
        // значит и оценка остатка совсем другая.
        n = (int) ((log(eps / 2.0) + log(fabs(x) - 2.0)) / log(2.0 / fabs(x))) + 1;
    }
    
    n = (n > 0) ? n : 1;
    n = (n > MAX_SERIES_N) ? MAX_SERIES_N : n;
    x = 2 / x;
    printf("Для обеспечения требуемой точности нужно n = %d членов ряда.\n", n);
    
    a = (n % 2 == 1) ? 0.5 : -0.5;
    b = a;
    for (i = 0; i < n - 1; ++i) {
        a = -a;
        b = a + b * x;
    }
    b = b * x;
    
    printf("Частичная сумма ряда S_%d = %lf\n", n, b);

    return 0;
}
\end{letiListing}

\newpage\begin{letiListing}[lst:constant-time]{Вычисление суммы ряда в замкнутой форме}{c}
#include <stdio.h>


int main() {
    double x;

    printf("S = (1 / x) - (2 / x^2) + (4 / x^3) - (8 / x^4) + ...\n");
    printf("x = ");
    scanf("%lf", &x);

    if (x >= -2.0 && x <= 2.0) {
        // Ряд расходится, программа завершается.
        printf("\nРяд расходится в заданной точке x.\n");
    }
    else {
        printf("Сумма ряда S = %lf\n", 1 / (x + 2));
    }

    return 0;
}
\end{letiListing}

\end{document}
